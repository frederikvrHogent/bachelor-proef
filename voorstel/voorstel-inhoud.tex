%---------- Inleiding ---------------------------------------------------------

\section{Introductie} % The \section*{} command stops section numbering
\label{sec:introductie}
In het laatste decenium is data-generatie en -verwerking steeds meer op de voorgrond getreden. Moderne bedrijven proberen zo veel mogelijk data digitaal te genereren en op te slaan - zonder dat daar fysieke documenten aan te pas komen. Een groot deel van de bedrijven heeft echter nog steeds een verouderd papieren archief.
Digitalisatie van documenten is van cruciaal belang voor een bedrijf om gemakkelijker de huidige stand van zaken te analyseren en om te proberen toekomstige trends te voorspellen(add reference). In verschillende sectoren wordt er daarom steeds meer ingezet op digitalisatie van deze papieren archieven (add reference). 

Binnen dit vakgebied wordt er veelal gefocused op het archiveren van documenten met tekst, er bestaat weinig onderzoek naar het archiveren van foto-archieven op basis van machine learning. Aangezien digitale foto's pas populairder werden dan fysieke foto's rond het jaar 2000 (add reference) zijn er ongetwijfeld nog archieven met niet-gedigitaliseerde foto's. Indien foto's wel al gedigitaliseerd werden ontbreekt er in veel gevallen metadata. Met metadata wordt begrepen dat er naast de afbeelding ook extra data wordt bijgehouden. In de context van dit onderzoek wordt met metadata steeds bedoeld dat er per foto woorden worden bijgehouden waarop text-searches uitgevoerd kunnen worden zodat de foto gemakkelijk kan gevonden worden.

Binnen het automatisch labelen van documenten is er door middel van machine learning grote vooruitgang geboekt (add reference). Een belangrijke evolutie is dat engines nu algemeen getrained worden en niet meer specifiek worden opgezet voor specifieke type's afbeeldingen. Deze moderne engines kunnen een breed aantal types afbeeldingen interpreteren en trachten correct te labelen. In dit onderzoek werd gekozen om enkel met machine learning engines te werken die algemeen getraind werden. Binnen deze subset van machine learning engines zijn er 4 grote spelers: Google Vision, AWS Rekognition, Microsoft Azure, IBM Watson. De IBM Watson engine focust zich volledig op face recognition en is daarom niet toepasbaar voor dit onderzoek. Google Vision, AWS Rekognition en Microsoft Azure omvatten alle drie label-detectie functionaliteiten, zijn uitvoerig gedocumenteerd en hebben een gratis tier zolang er onder de 1000 afbeeldingen per maand verwerkt worden. Om de scope beperkt te houden werd er gekozen om enkel Google Vision en AWS Rekognition te gebruiken.
Dit onderzoek zal aantonen welke engine het meest correct is voor het labelen van dieren foto's, naast correctheid van de voorspellingen zal er gekeken worden welke zekerheidsgraad de engines geven bij juiste en foute voorspellingen. Pricing models worden buiten beschouwing gelaten.

Omdat de scope foto's nog steeds zeer breed is zal er verder beperkt woren tot foto's van dieren. Foto's van dieren zijn een ideaal onderwerp voor machine learning labeling omdat het een categorie is met duizenden subcagetorien en er is een bijna oneindige bron sample-materiaal beschikbaar.


%---------- Stand van zaken ---------------------------------------------------

\section{State-of-the-art}
\label{sec:state-of-the-art}

Er is weinig wetenschappelijk onderzoek verricht naar de mate van correctheid van image labeling op basis van machine learning modellen. Eerder onderzoek focust zich op het labelen van images op basis van mining van zoekresultaten \autocite{Wang2008} of op het opstellen van grafische user-interfaces voor het labelen van images \autocite{LuisvonAhn2004} and \autocite{JuliaMoehrmann2012}.





% Voor literatuurverwijzingen zijn er twee belangrijke commando's:
% \autocite{KEY} => (Auteur, jaartal) Gebruik dit als de naam van de auteur
%   geen onderdeel is van de zin.
% \textcite{KEY} => Auteur (jaartal)  Gebruik dit als de auteursnaam wel een
%   functie heeft in de zin (bv. ``Uit onderzoek door Doll & Hill (1954) bleek
%   ...'')


%---------- Methodologie ------------------------------------------------------
\section{Methodologie}
\label{sec:methodologie}

Het vergelijkend onderzoek tussen deze twee engines wordt opgedeeld in drie onderdelen:

\begin{enumerate}
    \item Opstellen van test-data
    \item Versturen van de test-data naar de engines
    \item Analyseren van het resultaat
\end{enumerate}

1. Opstellen van test-data
\linebreak
Op basis van Google image searches wordt er een set van afbeeldingen gecreëerd, deze afbeeldingen worden in subcategorieen onderverdeeld naar soort dier.

2. Versturen an de test-data naar de engines
\linebreak
Zowel bij Google Vision als bij AWS Rekognition moet er eerst een account aangemaakt worden voor de image labeling gebruikt kan worden. Beide engines laten het toe om minstens 1000 afbeeldingen gratis te verwerken. De engines zijn bereikbaar via een API, die geïntegreerd kan worden met respectievelijk de .Net Google Cloud Vision Package en de AWS SDK.

Via een .Net console project zullen beide SDK's geintegreerd worden. De test-data wordt naar beide API's gestuurd, de resultaten worden opgevangen en in een csv file weggeschreven.

3. Analyseren van het resultaat
\linebreak
Op basis van deze de csv file wordt er een analyse gemaakt over de kwaliteit van beide engines.

Het is belangrijk om bij het analyseren van de resultaten niet alleen naar de succesvolle resultaten te bekijken maar ook te checken hoeveel resultaten er false positives zijn. False positivies zijn afbeedlingen waarvan de engine zeker is dat het resultaat juist is maar waarvan het resultaat verkeerd is.
De false postiive analyse zal gebeuren op basis van de zekerheidsgraad of confidence level die ook meegestuurd wordt door de API's.

Een laatste factor die zal meetellen in de analyse is de snelheid van verwerken, hoe snel konden de engines de afbeeldingen verwerken en een resultaat sturen. In dit onderzoek met beperkte data zal de snelheid van verwerking geen rol spelen, maar naar schaalbaarheid is dit een belangrijke factor.

%---------- Verwachte resultaten ----------------------------------------------
\section{Verwachte resultaten}
\label{sec:verwachte_resultaten}
De engines worden geacht om met een grote mate van zekerheid, minstens 80 percent, een correct label toe te herkennen voor de afbeelding. Een bepalende factor zal waarschijnlijk de hoeveelheid ruis zijn die op de afbeelding aanwezig is. Met ruis worden achtegrond-figuren of voorwerpen bedoeld die de herkenning verstoren.
Er wordt verwacht dat de verwerkingssnelheid heel hoog zal liggen, met verwerkingstijdens per afbeeldingen <1 seconde.

%---------- Verwachte conclusies ----------------------------------------------
\section{Verwachte conclusies}
\label{sec:verwachte_conclusies}
Aangezien beide engines door grote bedrijven worden gebacked met veel financiele middelen wordt er niet verwacht dat er een groot verschil zal zijn tussen het resultaat in de image processing. Vermoedelijk zal 1 engine beter zijn in een bepaald deelprobleem en zal de andere engine dan weer beter zijn in een ander deelprobleem.
Aangezien er geen wetenschappelijk onderzoek te vinden is die beide engines met elkaar vergelijkt zal er gewacht moet worden op het resultaat van het onderzoek om de conclusie beter te definieren.

