%==============================================================================
% Sjabloon onderzoeksvoorstel bachelorproef
%==============================================================================
% Gebaseerd op LaTeX-sjabloon ‘Stylish Article’ (zie voorstel.cls)
% Auteur: Jens Buysse, Bert Van Vreckem
%
% Compileren in TeXstudio:
%
% - Zorg dat Biber de bibliografie compileert (en niet Biblatex)
%   Options > Configure > Build > Default Bibliography Tool: "txs:///biber"
% - F5 om te compileren en het resultaat te bekijken.
% - Als de bibliografie niet zichtbaar is, probeer dan F5 - F8 - F5
%   Met F8 compileer je de bibliografie apart.
%
% Als je JabRef gebruikt voor het bijhouden van de bibliografie, zorg dan
% dat je in ``biblatex''-modus opslaat: File > Switch to BibLaTeX mode.

\documentclass{voorstel}

\usepackage{lipsum}

%------------------------------------------------------------------------------
% Metadata over het voorstel
%------------------------------------------------------------------------------

%---------- Titel & auteur ----------------------------------------------------

% TODO: geef werktitel van je eigen voorstel op
\PaperTitle{Vergelijking van image labeling van dierenfoto's met machine learning tussen Google Vision en AWS Rekognition}
\PaperType{Onderzoeksvoorstel Bachelorproef 2020-2021} % Type document

% TODO: vul je eigen naam in als auteur, geef ook je emailadres mee!
\Authors{Frederik Van Ruyskensvelde\textsuperscript{1}} % Authors
\CoPromotor{StaatNogNietVast\textsuperscript{2} (Bedrijfsnaam)}
\affiliation{\textbf{Contact:}
  \textsuperscript{1} \href{mailto:frederik.vanruyskensvelde@student.hogent.be}{frederik.vanruyskensvelde@student.hogent.be};
  \textsuperscript{2} \href{mailto:}{};
}

%---------- Abstract ----------------------------------------------------------

\Abstract{
    In deze bachelorproef wordt een korte vergelijkende studie uitgevoerd tussen twee computer vision engines die op basis van machine learning afbeeldingen taggen. De gekozen engines voor dit onderzoek zijn Google Vision en AWS Rekognition - beide worden aanzien als toonaangevend in het vakgebied (add reference). Binnen het kader van digitale archiving wordt onderzocht welke engine een hogere succes-rate heeft voor het labelen van afbeeldingen. De scope van dit onderzoek wordt beperkt tot foto's omdat er reeds een groot aantal studies zijn die computer vision voor geschreven documenten bestuderen (add reference). Binnen de scope foto's wordt er verder beperkt tot foto's van dieren, een ideale categorie om machine learning mee te testen aangezien er veel subcategorien zijn en er veel sample materiaal beschikbaar is.

    Tijdens het onderzoek zal er test-data aangeboden worden aan de engines via Google Cloud Vision API en Amazon Rekognition API, de resultaten worden geïnterpreteerd en met elkaar vergeleken. Het resultaat zal uitwijzen welke engine correcter werkt en of de succes-marge tussen beide API's even groot is. 
    Aangezien de machine learning engines steeds in volle evolutie zijn zal het resultaat van dit onderzoek slechts een snapshot zijn van de huidige kwaliteit van deze engines. Er wordt verwacht dat er in machine learning en computer vision op korte termijn grote vooruitgang zal geboekt worden (add reference) - deze studie kan als referentiewerk dienen voor toekomstig onderzoek.    
}

%---------- Onderzoeksdomein en sleutelwoorden --------------------------------
% TODO: Sleutelwoorden:
%
% Het eerste sleutelwoord beschrijft het onderzoeksdomein. Je kan kiezen uit
% deze lijst:
%
% - Mobiele applicatieontwikkeling
% - Webapplicatieontwikkeling
% - Applicatieontwikkeling (andere)
% - Systeembeheer
% - Netwerkbeheer
% - Mainframe
% - E-business
% - Databanken en big data
% - Machineleertechnieken en kunstmatige intelligentie
% - Andere (specifieer)
%
% De andere sleutelwoorden zijn vrij te kiezen

\Keywords{Machineleertechnieken en kunstmatige intelligentie. Computer vision --- Machine learning --- Data-labeling} % Keywords
\newcommand{\keywordname}{Sleutelwoorden} % Defines the keywords heading name

%---------- Titel, inhoud -----------------------------------------------------

\begin{document}

\flushbottom % Makes all text pages the same height
\maketitle % Print the title and abstract box
\tableofcontents % Print the contents section
\thispagestyle{empty} % Removes page numbering from the first page

%------------------------------------------------------------------------------
% Hoofdtekst
%------------------------------------------------------------------------------

% De hoofdtekst van het voorstel zit in een apart bestand, zodat het makkelijk
% kan opgenomen worden in de bijlagen van de bachelorproef zelf.
%---------- Inleiding ---------------------------------------------------------

\section{Introductie} % The \section*{} command stops section numbering
\label{sec:introductie}
In het laatste decenium is data-generatie en -verwerking steeds meer op de voorgrond getreden. Moderne ondernemingen proberen zo veel mogelijk data digitaal te genereren en op te slaan - zonder dat daar fysieke documenten aan te pas komen. Een groot deel van de ondernemingen heeft echter naast het digitaal archief nog steeds een papieren archief.
Digitalisatie van documenten is van cruciaal belang om gemakkelijker de huidige stand van zaken te analyseren en om toekomstige trends te voorspellen. In verschillende sectoren wordt er daarom steeds meer ingezet op digitalisatie van deze papieren archieven \autocite{PekkaLeviaekangas2020}. 

Binnen het vakgebied digitale archiveringen wordt er veelal gefocused op het archiveren van documenten met tekst op basis van OCR (Optical Character Recognition). Met OCR wordt geprobeerd om tekst op afbeeldingen om te zetten naar tekst in string vorm. Aangezien de meeste foto's geen tekst bevatten die nuttig zijn voor het categoriseren van de foto kunnen de OCR technieken niet op dit vakgebied toegepast worden.

Digitale foto's werden pas populairder dan fysieke foto's rond het jaar 2000 \autocite{HenryC.LucasJr.2009} - er zijn dus ongewtijfeld ondernemingen die met niet-gedigitaliseerde foto-archieven zitten. Indien foto's wel al gedigitaliseerd werden, voornamelijk via scans, ontbreekt er in veel gevallen metadata. Met metadata wordt begrepen dat er naast de afbeelding ook extra data wordt bijgehouden. In de context van deze Bachelorproef wordt met metadata steeds bedoeld dat er per foto woorden worden bijgehouden waarop text-searches uitgevoerd kunnen worden zodat de foto gemakkelijk kan gevonden worden. Het toevoegen van metadata noemt men ook indexeren.

Zoals aangegeven kunnen de gebruikelijke OCR technieken niet gebruikt worden voor afbeeldingen, daarom wordt er in dit onderzoek gefocused op het gebruiken van machine learning engines om afbeeldingen te indexeren. Een belangrijke evolutie is dat er nu engines bestaan die algemeen getrained worden en niet meer specifiek worden opgezet voor specifieke type's afbeeldingen. Deze moderne engines kunnen een breed aantal type's afbeeldingen interpreteren en trachten correct te indexeren.
In dit onderzoek werd bijgevolg gekozen om enkel met machine learning engines te werken die algemeen getraind werden. Binnen deze subset van machine learning engines zijn er 4 grote spelers: Google Vision, AWS Rekognition, Microsoft Azure, IBM Watson. De IBM Watson engine focust zich volledig op face recognition en is daarom niet toepasbaar voor dit onderzoek. Google Vision, AWS Rekognition en Microsoft Azure omvatten alle drie brede indexatie functionaliteiten, zijn uitvoerig gedocumenteerd en hebben een gratis tier zolang er onder de 1000 afbeeldingen per maand verwerkt worden. Om de scope beperkt te houden werd er gekozen om enkel Google Vision en AWS Rekognition te gebruiken.
Dit onderzoek zal aantonen welke engine het meest correct is voor het indexeren van dieren foto's, naast correctheid van de indexatie zal er gekeken worden welke zekerheidsgraad de engines geven bij juiste en foute voorspellingen. Pricing models worden buiten beschouwing gelaten.

Omdat de scope foto's zonder belangrijke tekst nog steeds zeer breed is zal er verder beperkt woren tot foto's van dieren. Foto's van dieren zijn een ideaal onderwerp voor machine learning indexatie omdat het een categorie is met duidelijk afgebakende subcategorien per diersoort en er een onuitputbare bron gratis sample materiaal beschikbaar is op het internet.

%---------- Stand van zaken ---------------------------------------------------

\section{State-of-the-art}
\label{sec:state-of-the-art}

Er is weinig wetenschappelijk onderzoek verricht naar de mate van correctheid van indexatie van afbeeldingen op basis van machine learning engines. Eerder onderzoek focust zich op het labelen van images op basis van mining van zoekresultaten \autocite{Wang2008} of op het opstellen van grafische user-interfaces voor het indexeren van images \autocite{LuisvonAhn2004} en \autocite{JuliaMoehrmann2012}.

Het feit dat er weinig onderzoek is kan een aanwijzing zijn hoe indexatie binnen digital archiving momenteel gebeurt: gebruikers kijken iedere foto manueel na en voegen metadata toe of laten de afbeeldingen zonder metadata (en dus nagenoeg onvindbaar) in de archieven staan.
Ook foto's van wildcamera's, belangrijk voor het beheer van het dierenbestand, worden case by case bekeken en wordt er manueel metadata toegevoegd.

Manueel metadata toevoegen, of ook manueel indexeren, is tijdrovend en weinig intellectueel uitdagend werk - uit dit onderzoek zal tenminste deels duidelijk worden of machine learning engines gebruikt zouden kunnen worden om op automatische wijze afbeeldingen te indexeren.


% Voor literatuurverwijzingen zijn er twee belangrijke commando's:
% \autocite{KEY} => (Auteur, jaartal) Gebruik dit als de naam van de auteur
%   geen onderdeel is van de zin.
% \textcite{KEY} => Auteur (jaartal)  Gebruik dit als de auteursnaam wel een
%   functie heeft in de zin (bv. ``Uit onderzoek door Doll & Hill (1954) bleek
%   ...'')


%---------- Methodologie ------------------------------------------------------
\section{Methodologie}
\label{sec:methodologie}

Het vergelijkend onderzoek tussen Google Vision en AWS Rekognition wordt opgedeeld in drie onderdelen:

\begin{enumerate}
    \item Opstellen van test-data
    \item Versturen van de test-data naar de engines
    \item Analyseren van het resultaat
\end{enumerate}

1. Opstellen van test-data
\linebreak
Op basis van Google image searches wordt er een set van afbeeldingen gecreëerd, deze afbeeldingen worden in subcategorieen onderverdeeld naar soort dier.
Opslaan van de afbeeldingen gebeurt eerst lokaal, vervolgens worden de afbeeldingen per engine klaar gezet in zip format voor Google Vision en een AWS S3 Bucket voor AWS Rekognition.

2. Versturen van de test-data naar de engines
\linebreak
Zowel bij Google Vision als bij AWS Rekognition moet er eerst een account aangemaakt worden voor de image labeling gebruikt kan worden. Beide engines laten het toe om minstens 1000 afbeeldingen  per maandgratis te verwerken. De engines zijn bereikbaar via een API, die geïntegreerd kan worden met respectievelijk de .Net Google Cloud Vision Package en de AWS SDK.

Via een .Net console project zullen beide SDK's geintegreerd worden. De test-data wordt naar beide API's gestuurd, de resultaten worden opgevangen en in een csv file weggeschreven.

3. Analyseren van het resultaat
\linebreak
Op basis van de csv file's wordt er een brede analyse gemaakt van de resultaten voor beide engines.

Het is belangrijk om bij het analyseren van de resultaten niet alleen naar de correctheid te kijken van de indexatie maar ook te checken hoeveel resultaten er false positives zijn. False positivies zijn afbeeldingen waarvan de engine zeker is dat de indexatie correct is maar waarvan de indexatie verkeerd is.
De false postiive analyse zal gebeuren op basis van de zekerheidsgraad of confidence level die per indexatie ook meegestuurd wordt door de API's. De zekerheidgraad wordt uitgedrukt in een percentuele waarde. We zullen de zekerheidsgraden indelen in 3 categoriën:

\begin{enumerate}
    \item Onder 60 percent zekerheid: Lage zekerheidsgraad
    \item Tussen 60 en 80 percent: Gemiddelde zekerheidsgraad
    \item Boven 80 percent: Hoge zekerheidsgraad
\end{enumerate}

Een laatste factor die zal meetellen in de analyse is de snelheid van verwerken, hoe snel konden de engines de afbeeldingen verwerken en een resultaat sturen. In dit onderzoek met beperkte data zal de snelheid van verwerking geen rol spelen, maar naar schaalbaarheid is dit een belangrijke factor.
Snelheid zal bijgehouden worden tot op de milliseconde, om te vermijden dat een tijdelijke storing voor oneerlijke vertraging zorgt zal het internet tijdens verwerking steeds via ethernet verbonden zijn en wordt het gemiddelde berekend op 2 test-runs van dezelfde data.

%---------- Verwachte resultaten ----------------------------------------------
\section{Verwachte resultaten}
\label{sec:verwachte_resultaten}
De engines worden geacht om afbeeldingen van dieren ten minstens 80 percent correct kunnen indexeren met een hoge zekerheidsgraad. Een bepalende factor voor correcte indexatie zal waarschijnlijk de hoeveelheid ruis zijn die op de afbeelding aanwezig is. Met ruis worden achtegrond-figuren of voorwerpen bedoeld die de herkenning verstoren.
Er wordt verwacht dat de gemiddelde verwerkingssnelheid per afbeelding minder dan 1 seconde zal zijn beide engines.
Bij afbeeldingen waarvan de zekerheidsgraag laag is wordt er verwacht dat er 50 percent afbeeldingen een false positive zullen opleveren.

%---------- Verwachte conclusies ----------------------------------------------
\section{Verwachte conclusies}
\label{sec:verwachte_conclusies}
Er wordt niet verwacht dat er een significant verschil zal zijn tussen het resultaat van indexatie bij beide engines. Vermoedelijk zal 1 engine beter zijn in een bepaald deelprobleem en zal de andere engine dan weer beter zijn in een ander deelprobleem.
Aangezien er geen wetenschappelijk onderzoek te vinden is die beide engines met elkaar vergelijkt zal er gewacht moet worden op het resultaat van dit onderzoek om de conclusie beter te definieren.
Uit de vergelijkende studie zal duidelijk worden welke engine er het beste gebruikt kan worden voor indexeren van foto-archieven. Indien er een hoge zekerheidsgraad is met weinig false positives zal de conclusie getrokken worden dat machine learning engines een nuttige tool kunnen zijn voor archivarissen.



%------------------------------------------------------------------------------
% Referentielijst
%------------------------------------------------------------------------------
% TODO: de gerefereerde werken moeten in BibTeX-bestand ``voorstel.bib''
% voorkomen. Gebruik JabRef om je bibliografie bij te houden en vergeet niet
% om compatibiliteit met Biber/BibLaTeX aan te zetten (File > Switch to
% BibLaTeX mode)

\phantomsection
\printbibliography[heading=bibintoc]

\end{document}
