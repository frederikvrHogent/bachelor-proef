%==============================================================================
% Sjabloon onderzoeksvoorstel bachelorproef
%==============================================================================
% Gebaseerd op LaTeX-sjabloon ‘Stylish Article’ (zie voorstel.cls)
% Auteur: Jens Buysse, Bert Van Vreckem
%
% Compileren in TeXstudio:
%
% - Zorg dat Biber de bibliografie compileert (en niet Biblatex)
%   Options > Configure > Build > Default Bibliography Tool: "txs:///biber"
% - F5 om te compileren en het resultaat te bekijken.
% - Als de bibliografie niet zichtbaar is, probeer dan F5 - F8 - F5
%   Met F8 compileer je de bibliografie apart.
%
% Als je JabRef gebruikt voor het bijhouden van de bibliografie, zorg dan
% dat je in ``biblatex''-modus opslaat: File > Switch to BibLaTeX mode.\cite{ReaganMoore2000}

\documentclass{voorstel}

\usepackage{lipsum}

%------------------------------------------------------------------------------
% Metadata over het voorstel
%------------------------------------------------------------------------------

%---------- Titel & auteur ----------------------------------------------------

% TODO: geef werktitel van je eigen voorstel op
\PaperTitle{Vergelijking van image labeling van dierenfoto's met machine learning tussen Google Vision en AWS Rekognition}
\PaperType{Onderzoeksvoorstel Bachelorproef 2020-2021} % Type document

% TODO: vul je eigen naam in als auteur, geef ook je emailadres mee!
\Authors{Frederik Van Ruyskensvelde\textsuperscript{1}} % Authors
\CoPromotor{StaatNogNietVast\textsuperscript{2} (Bedrijfsnaam)}
\affiliation{\textbf{Contact:}
  \textsuperscript{1} \href{mailto:frederik.vanruyskensvelde@student.hogent.be}{frederik.vanruyskensvelde@student.hogent.be};
  \textsuperscript{2} \href{mailto:}{};
}

%---------- Abstract ----------------------------------------------------------

\Abstract{
    In deze bachelorproef wordt een  vergelijkende studie uitgevoerd tussen twee computer vision engines die op basis van machine learning afbeeldingen labelen. De gekozen engines voor dit onderzoek zijn Google Vision en AWS Rekognition. De scope van dit onderzoek wordt beperkt tot foto's omdat er reeds een groot aantal studies zijn die computer vision voor geschreven documenten bestuderen. Binnen de scope foto's wordt er verder beperkt tot foto's van dieren.

    Tijdens het onderzoek zal er test-data aangeboden worden aan de engines via Google Cloud Vision API en Amazon Rekognition API, de resultaten worden geïnterpreteerd en met elkaar vergeleken. Het resultaat zal uitwijzen welke engine de meeste correcte labeling uitvoert. 
    Aangezien de machine learning engines steeds in volle evolutie zijn zal het resultaat van dit onderzoek slechts een snapshot zijn van de huidige kwaliteit van deze engines - deze studie kan als referentiewerk dienen voor toekomstig onderzoek.    
}

%---------- Onderzoeksdomein en sleutelwoorden --------------------------------
% TODO: Sleutelwoorden:
%
% Het eerste sleutelwoord beschrijft het onderzoeksdomein. Je kan kiezen uit
% deze lijst:
%
% - Mobiele applicatieontwikkeling
% - Webapplicatieontwikkeling
% - Applicatieontwikkeling (andere)
% - Systeembeheer
% - Netwerkbeheer
% - Mainframe
% - E-business
% - Databanken en big data
% - Machineleertechnieken en kunstmatige intelligentie
% - Andere (specifieer)
%
% De andere sleutelwoorden zijn vrij te kiezen

\Keywords{Machineleertechnieken en kunstmatige intelligentie. Computer vision --- Machine learning --- Data-labeling} % Keywords
\newcommand{\keywordname}{Sleutelwoorden} % Defines the keywords heading name

%---------- Titel, inhoud -----------------------------------------------------

\begin{document}

\flushbottom % Makes all text pages the same height
\maketitle % Print the title and abstract box
\tableofcontents % Print the contents section
\thispagestyle{empty} % Removes page numbering from the first page

%------------------------------------------------------------------------------
% Hoofdtekst
%------------------------------------------------------------------------------

% De hoofdtekst van het voorstel zit in een apart bestand, zodat het makkelijk
% kan opgenomen worden in de bijlagen van de bachelorproef zelf.
%---------- Inleiding ---------------------------------------------------------

\section{Introductie} % The \section*{} command stops section numbering
\label{sec:introductie}
In het laatste decenium is data-generatie en -verwerking steeds meer op de voorgrond getreden. Moderne bedrijven proberen zo veel mogelijk data digitaal te genereren en op te slaan - zonder dat daar fysieke documenten aan te pas komen. Een groot deel van de bedrijven heeft echter nog steeds een verouderd papieren archief.
Digitalisatie van documenten is van cruciaal belang voor een bedrijf om gemakkelijker de huidige stand van zaken te analyseren en om te proberen toekomstige trends te voorspellen(add reference). In verschillende sectoren wordt er daarom steeds meer ingezet op digitalisatie van deze papieren archieven (add reference). 

Binnen dit vakgebied wordt er veelal gefocused op het archiveren van documenten met tekst, er bestaat weinig onderzoek naar het archiveren van foto-archieven op basis van machine learning. Aangezien digitale foto's pas populairder werden dan fysieke foto's rond het jaar 2000 (add reference) zijn er ongetwijfeld nog archieven met niet-gedigitaliseerde foto's. Indien foto's wel al gedigitaliseerd werden ontbreekt er in veel gevallen labeling - zonder labeling is de foto onvindbaar bij zoekopdrachten.

Binnen het automatisch labelen van documenten is er door middel van machine learning grote vooruitgang geboekt (add reference). Een belangrijke evolutie is dat engines nu algemeen getrained worden en niet meer specifiek worden opgezet voor specifieke type's afbeeldingen. Deze moderne engines kunnen voor een breed aantal types afbeeldingen juiste herkenningen maken. In dit onderzoek werd gekozen om enkel met machine learning engines te werken die algemeen getraind werden, de meest toonaangevende engines (add reference) die op deze manier werken zijn Google Vision en AWS Rekognition.
Dit onderzoek zal aantonen welke engine de beste performance heeft, pricing models worden buiten beschouwing gelaten.

Omdat de scope foto's nog steeds zeer breed is zal er verder beperkt woren tot foto's van dieren. Foto's van dieren zijn een ideaal onderwerp voor machine learning labeling omdat het een categorie is met veel subcagetorien en er is een grote hoeveelheid sample-materiaal beschikbaar.


%---------- Stand van zaken ---------------------------------------------------

\section{State-of-the-art}
\label{sec:state-of-the-art}

Er is weinig wetenschappelijk onderzoek verricht naar image labeling op basis van machine learning modellen. Eerder onderzoek focust zich meer op het labelen van images op basis van mining van zoekresultaten (https://ieeexplore.ieee.org/abstract/document/4527251) of op het opstellen van grafische user-interfaces voor het labelen van images (https://dl.acm.org/doi/abs/10.1145/2304496.2304498) (https://dl.acm.org/doi/abs/10.1145/985692.985733).

% Voor literatuurverwijzingen zijn er twee belangrijke commando's:
% \autocite{KEY} => (Auteur, jaartal) Gebruik dit als de naam van de auteur
%   geen onderdeel is van de zin.
% \textcite{KEY} => Auteur (jaartal)  Gebruik dit als de auteursnaam wel een
%   functie heeft in de zin (bv. ``Uit onderzoek door Doll & Hill (1954) bleek
%   ...'')


%---------- Methodologie ------------------------------------------------------
\section{Methodologie}
\label{sec:methodologie}

Het vergelijkend onderzoek tussen deze twee engines wordt opgedeeld in drie onderdelen:

\begin{enumerate}
    \item Opstellen van test-data
    \item Versturen van de test-data naar de engines
    \item Analyseren van het resultaat
\end{enumerate}

1. Opstellen van test-data
\linebreak
Op basis van Google image searches wordt er een set van afbeeldingen gecreëerd, deze afbeeldingen worden in subcategorieen onderverdeeld naar soort dier.

2. Versturen an de test-data naar de engines
\linebreak
Zowel bij Google Vision als bij AWS Rekognition moet er eerst een account aangemaakt worden voor de image labeling gebruikt kan worden. Beide engines laten het toe om minstens 1000 afbeeldingen gratis te verwerken. De engines zijn bereikbaar via een API, die geïntegreerd kan worden met een SDK.

Via een .net console project zullen beide SDK's geintegreerd worden. De test-data wordt naar beide API's gestuurd, de resultaten worden opgevangen en in een csv file weggeschreven.

3. Analyseren van het resultaat
\linebreak
Op basis van deze de csv file wordt er een analyse gemaakt over de kwaliteit van beide engines.

Het is belangrijk om bij het analyseren van de resultaten niet alleen naar de succesvolle resultaten te bekijken maar ook te checken hoeveel resultaten er false positives zijn. False positivies zijn afbeedlingen waarvan de engine zeker is dat het resultaat juist is maar waarvan het resultaat verkeerd is.
De false postiive analyse zal gebeuren op basis van de zekerheidsgraad of confidence level die ook meegestuurd wordt door de API's.

Een laatste factor die zal meetellen in de analyse is de snelheid van verwerken, hoe snel konden de engines de afbeeldingen verwerken en een resultaat sturen. In dit onderzoek met beperkte data zal de snelheid van verwerking geen rol spelen, maar naar schaalbaarheid is dit een belangrijke factor.

%---------- Verwachte resultaten ----------------------------------------------
\section{Verwachte resultaten}
\label{sec:verwachte_resultaten}
De engines worden geacht om met een grote mate van zekerheid, minstens 80 percent, correct afbeeldingen te analyseren en te labelen. Een bepalende factor zal waarschijnlijk de hoeveelheid ruis zijn die op de afbeelding aanwezig is. Met ruis worden achtegrond-figuren of voorwerpen bedoeld die de herkenning verstoren.
Er wordt verwacht dat de verwerkingssnelheid heel hoog zal liggen, met verwerkingstijdens per afbeeldingen <1 seconde.

%---------- Verwachte conclusies ----------------------------------------------
\section{Verwachte conclusies}
\label{sec:verwachte_conclusies}
Aangezien beide engines door grote bedrijven worden gebacked met veel financiele middelen wordt er niet verwacht dat er een groot verschil zal zijn tussen het resultaat in de image processing. Vermoedelijk zal 1 engine beter zijn in een bepaald deelprobleem en zal de andere engine dan weer beter zijn in een ander deelprobleem.
Aangezien er geen wetenschappelijk onderzoek te vinden is die beide engines met elkaar vergelijkt zal er gewacht moet worden op het resultaat van het onderzoek om de conclusie beter te definieren.



%------------------------------------------------------------------------------
% Referentielijst
%------------------------------------------------------------------------------
% TODO: de gerefereerde werken moeten in BibTeX-bestand ``voorstel.bib''
% voorkomen. Gebruik JabRef om je bibliografie bij te houden en vergeet niet
% om compatibiliteit met Biber/BibLaTeX aan te zetten (File > Switch to
% BibLaTeX mode)

\phantomsection
\printbibliography

\end{document}
