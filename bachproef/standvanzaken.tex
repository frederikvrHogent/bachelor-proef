\chapter{\IfLanguageName{dutch}{Stand van zaken}{State of the art}}
\label{ch:stand-van-zaken}

% Tip: Begin elk hoofdstuk met een paragraaf inleiding die beschrijft hoe
% dit hoofdstuk past binnen het geheel van de bachelorproef. Geef in het
% bijzonder aan wat de link is met het vorige en volgende hoofdstuk.

% Pas na deze inleidende paragraaf komt de eerste sectiehoofding.
In dit hoofdstuk wordt besproken wat machine learning inhoudt en wat de belangrijskte stromingen zijn. Vervolgens wordt er dieper ingegaan op het classificeren van afbeeldingen; wat houdt deze term in en hoe wordt dit uitgevoerd door mens en machine. Aangezien dit onderzoek zich focust op het nut van machine learning modellen voor het archiveren van foto's komen ook de belangrijkste stromingen in archiveren van foto's naar voor.

\section{\IfLanguageName{dutch}{Machine learning}{machine learning}}
\label{sec:machine-learning}
Machine learning is een bepaalde soort computer algoritme dat ontworpen werd om menselijke intelligentie te emuleren door bij te leren uit de directe omgeving \autocite{ElNaqa2015}. Op basis van vroegere ervaringen zal het algoritme trachten om efficientie te verhogen of correcte voorspellingen te maken. In deze context wordt ervaring gedefinieerd als informatie die voor het systeem beschikbaar is, typisch in de vorm van elektronische data. Deze elektronische data kan verschillende vormen aannemen, bijvoorbeed datasets gelabeld door mensen of data gegeneerd door andere machines op basis van hun contact met de buitenwereld. De hoeveelheid en kwaliteit van de data zal steeds cruciaal zijn voor de performantie van het machine learning algoritme \autocite{Mohri2018}.

Machine learning algoritmes bieden computer de mogelijkheid om te leren zonder expliciet geprogrammeerd te zijn \autocite{DeVreese2017}. De problemen waarvoor machine learning meestal gebruikt wordt zijn te complex om via klassieke programmeertechnieken aan te pakken. Aangezien machine learning slechts een techniek is waarmee men tracht complexe problemen op te lossen werd het succesvol gebruikt in uiteenlopende onderzoeksvelden zoals ruimtevaart, statistiek, biomedica en computer vision.

Machine learning algoritmes kunnen in het algmeen opgedeeld worden in 3 type's: supervised learning, unsupervised learning, reinforcement learning.

\subsection{\IfLanguageName{dutch}{Supervised learning}{Supervised learning}}
\label{sec:supervised-learning}
De belangrijskte karakteristiek van supervised learning is dat het machine learning algoritme een vooraf gelabelde trainingsdataset ter beschikking heeft. Supervised learning zal op basis van deze trainingsdataset trachten, door middel van inductie, modellen op te bouwen die gebruikt kunnen worden om andere ongelabelde datasets te classificeren. Achteraf kan dan gecontroleerd worden hoeveel percent van de nieuwe labels correct werd gevonden en hoe zeker het model was van deze voorspellingen. Zoals de naam van het model aangeeft is er een vorm van een 'supervisor' nodig die het model toont welke labels met welke ervaringen uit de trainingsdata moeten geassocieerd worden \autocite{Cunningham2008}. 

Een voorbeeld van supervised learning is hoe Netflix nieuwe content aan de users voorstelt. Het algoritme start van een gelabelde dataset met daarin de ervaringen van de gebruiker, op basis van deze ervaringen zal het model voorspellingen over nieuwe content maken.

\subsection{\IfLanguageName{dutch}{Unsupervised learning}{Unsupervised learning}}
\label{sec:unsupervised-learning}
In tegenstelling tot supervised learning wordt er bij unsupervised learning vertrokken van een ongelabelde dataset. Het doel is dat het algoritme deze ongelabelde dataset zal intepreteren en een interne representatie van de 'wereld' - de 'wereld' is alle informatie binnen de dataset - zal opbouwen. Een unsupervised learning model zal proberen een diepe kennis van de wereld op te bouwen aan de hand van complexe patroonherkenning \autocite{Hinton1999}. Op basis van deze diepere kennis zal het algoritme de data in verschillende clusters onderverdelen. Analyse van deze clusters kan dan leiden tot het labellen of classificeren van de ongelabelde dataset.

Unsupervised learning wordt in verschillende bedrijfstakken actief toegepast, Google News gebruikt bijvoorbeeld unsupervised learning om nieuwsartikelen rond hetzelfde nieuwsfeit automatisch te clusteren en op een georganiseerde manier aan de gebruiker aan te bieden.

\subsection{\IfLanguageName{dutch}{Reinforcement learning}{Reinforcement learning}}
\label{sec:reinforcement-learning}
Reinforcement learning is een machine learning algoritme dat steeds zal zoeken naar een zo hoog mogelijke beloning, uitgedrukt in een numerieke waarde. Het algoritme krijgt op voorhand geen uitleg wat goed of fout is maar moet zelf ontdekken welke acties leiden tot de grootst mogelijke beloning \autocite{Sutton2018}. Uit het iteratief uitvoeren van dezelfde opdracht en ontvangen van een beloning zal het algoritme een duidelijk beeld krijgen welke acties de grootste beloning opleveren. Het zal steeds een balans moeten houden tussen exploratief onderzoeken en in de diepte onderzoeken van reeds gevonden oplossingen. 

Reinforcement learning vindt toepassingen in verschillende sectoren zoals autonoom auto-rijden, handelen op de aandelenmarkt en gaming. Reinforcement learning sluit inherent dicht aan bij gaming dat met dezelfde soort trial-and-error beloningssystemen werkt, bijgevolg werd er reeds veel onderzoek verricht naar het gebruik van reinforcement learning om spelletjes te spelen. Een voorbeeld uit 2015 is een wedstrijd van het complexe bordspel Go tussen wereldkampioen Fan Hui en AlphaGo. AlphaGo is een algoritme ontwikkeld door Google, gebaseerd op het reinforcement learning principe. Fan Hui verloor de match met 1-4, het was de eerste keer dat een computer-gestuurd programma een Go wereldkampioen versloeg.

\section{\IfLanguageName{dutch}{Classificeren van afbeeldingen}{Image classification}}
\label{sec:classificeren-van-afbeeldingen}
In deze stand van zaken werd beknopt besproken wat machine learning is en wat de belangrijkste onderdelen zijn. De focus van deze bachelorproef is hoe deze machine learning algoritmes toegepast kunnen worden om afbeeldingen te classificeren. Het is dan ook belangrijk om verder te bespreken wat er bedoelt wordt met classificeren van afbeeldingen en wat de belangrijskte onderdelen zijn.

Het classificeren van afbeeldingen kan gedefinieerd worden als het groeperen van afbeeldingen in semantisch betekenisvolle categorien op basis van onderdelen van de afbeeldingen \autocite{VAILAYA19981921}. Deze ''semantisch betekenisvolle categorien'' worden ook labels genoemd, voorbeelden van labels zijn 'fiets', 'hond', 'boom'. Het correct aanduiden van labels is een belangrijk probleem wanneer men grote datasets van afbeeldingen heeft en deze zoekbaar wilt maken, bijvoorbeeld foto-archieven. Deze labels worden typisch niet in de afbeelding zelf opgeslaan maar in een bijhorend metadata bestand; wanneer er metadata voor een afbeelding beschikbaar is spreekt men van een verrijkte afbeelding.

Naast het zoeken van labels voor een afbeelding kan men ook trachten eventuele tekst op een afbeelding te lezen en deze toe te voegen aan de metadata, opnieuw wordt op deze manier de afbeelding verrijkt. Bijvoorbeeld het woord 'Politie' bij een foto van een politie-kantoor.

In de volgende secties worden er verschillende manieren besproken om afbeeldingen te classificeren.

%Een deel van het opgroeien is patronen herkennen in de wereld. Een mens opgroeit leert hij praten door de klanken na te bootsen rond hem, vervolgens leert hij bepaalde klanken met bepaalde fenomenen rond zich te identificeren. Bij het ouder worden zal een kind steeds beter worden in het herkennen van bepaalde begrippen door middel van trial and error. Het is een vorm van reinforcement learning. Een peuter leert het woordje bus, wanneer een peuter naar een bus wijst en 'bus' zegt zullen de ouders positief reageren. Wanneer het kind echter een vergissing maakt en naar een vrachtwagen wijst zullen de ouders het kind verbeteren. Op deze manier leren we vanop jonge leeftijd de wereld rond ons 'classificeren'.

\subsection{\IfLanguageName{dutch}{Manuele classificatie}{Manual classification}}
\label{sec:manuele-classificatie}
%Het manueel classificeren van afbeeldingen is een tijdrovend proces. Iedere afbeelding moet door een persoon geopend worden, deze persoon bekijkt de afbeelding en verrijkt de afbeelding aan de hand van labels. Voor het classificeren van miljoenen of miljarden afbeeldingen is manuele classificatie te tijdsintensief en wordt er naar automatische opties gezocht. Automatische of computergestuurde modellen hebben echter steeds manueel gelabelde datasets als trainingsmateriaal dus is het cruciaal dat de kwaliteit van deze manueel geclassificeerde datasets hoog is \autocite{JuliaMoehrmann2012}.
%
%Er werd reeds veel onderzoek en ontwikkeling verricht naar hoe men mensen kan helpen afbeeldingen op een snelle en correcte manier te labelen. Bijvoorbeeld via geavanceerde interfaces \autocite{JuliaMoehrmann2012} of gamification \autocite{LuisvonAhn2004}. Een voorbeeld van manuele classificatie is het invullen van een reCAPTCHA, de manuele input van deze beveiligingstechnologie wordt door Google gebruikt om machine learning modellen te trainen \autocite{Google2021}.

\subsection{\IfLanguageName{dutch}{Automatiche classificatie}{Automatic classification}}
\label{sec:automatic-classification}
Onder automatische classificatie begrijpen we het toewijzen van labels aan een afbeelding zonder menselijke input. Een computermodel krijgt een afbeelding aangeboden en wijst aan deze afbeelding verschillende labels toe. Aan iedere label zal een zekerheidsgraad meegegeven worden, uitgedrukt in procent.

Automatische classificatie gebeurt meestal aan de hand van machine learning modellen.

\subsubsection{\IfLanguageName{dutch}{Computer vision}{Computer vision}}
\label{sec:computer-vision}
Computer vision is de stroming binnen machine learning waarbij men tracht menselijk zicht na te bootsen. Computer vision bevindt zich in machine learning binnen supervised learning, op basis van een pregelabelde dataset worden voorspellingen over toekomstige afbeeldingen gemaakt.
Binnen de supervised learning subset wordt computer vision gedefinieerd als een deep learning model. Deep learning betekent dat er gebruikt wordt gemaakt van een neuraal netwerk met ten minste 3 of meer lagen.

- add explanation what is neural network -

Mensen herkennen afbeeldingen doordat hun ogen het beeld doorgeven naar de visuele cortex die het resultaat intepreteert. -> add reference
Door middel van miljoenen jaren evolutie is ons brein 'getrained' in het herkennen van objecten.

Voor computers is het echter veel moeilijker om te herkennen wat er op een afbeelding staat. Voor een computer ziet een afbeelding er typisch zo uit:
(add image bitmap) een lijst van integers. De computer weet hoe het deze integers moet omvormen tot een afbeelding, maar heeft hier geen informatie over de effectieve inhoud van de afbeelding.
De kunst van computers leren hoe ze een afbeelding kunnen intepreteren heet computer vision. 
Het belangrijkste dat computer vision nodig heeft is context. Net zoals bij een menselijk brein is context de bepalende factor voor het herkennen van afbeeldingen.
Op basis van machine learning worden er modellen opgezet die een computer context kunnen verschaffen bij de lange lijst nummers.

-> add explanation, images only make sense when we have some kind of context. eg an alien would not be able to 'understand' a picture taken on earth.

Wanneer men genoeg data verschaft aan algortime zal de computer op basis van deze context kunnen bepalen wat er op een afbeelding staat. Stel dat we een nieuw computer vision model trainen en we beginnen met foto's van tijgers en katten met tijgerpatroon. De computer zal na verloop van tijd eigenschappen herkennen die enkel bij tijgers of katten voorkomen en zo met een hoge accuraatheid voorspellingen kunnen maken. Sommige objecten zijn voor een mens moeilijk te onderscheiden, bijvoorbeeld wanneer men enkel de strepen van een dier te zien krijgt. Voor de computer kan echter op basis van data heel precies berekenen hoe groot de strepen zijn per diersoort en kan daarom wel gemakkelijk herkennen welk dier dit is.

Bij  het labelen van afbeeldingen kan men ook Optical Character Recognition (OCR) uitvoeren. Deze OCR engines zullen proberen tekst te herkennen op een afbeelding net zoals een mens dat doet. De OCR data kan gebruikt worden 

\subsubsection{\IfLanguageName{dutch}{Convolutional neural network}{Convolutional neural network}}
\label{sec:convolutional-neural-network}
Achterliggend zal bij computer vision steeds een Convolutional Neural Network (CNN) gebruikt worden. CNN werkt door afbeeldingen op te delen in kleinere groepjes pixels genaamd filters. Een CNN zal deze filters vergelijken met patronen die het reeds in de gelabelde trainingsdata heeft en zal proberen de afbeelding onder te delen in kleinere vormen. Bijvoorbeeld in de eerste laag van de CNN zal het algoritme grove verschillen herkennen, bijvoorbeeld randen en curves.
Hoe dieper het CNN gaat zal het uiteindelijk beginnen bepaalde objecten beginnen herkennen.

Het CNN zal beginnen met randomized filterwaardes en deze afchecken tegen de gelablede trainingsdata. Het afchecken geeft aan het CNN een bepaalde error-waarde (ofwel loss function) terug, deze error-waarde geeft aan hoe goed de voorspelling is. Op basis van deze error-waarde zal het algoritme bij iedere iteratie de filtervalues updaten. Idealiter zal iedere iteratie een iets grotere accuraatheid hebben. Na verloop van tijd zijn de filterwaardes goed genoeg zodat het CNN bepaalde zaken kan beginnen herkennen.

Om verder te gaan met het voorbeeld van een kat zijn dit enkele 'objecten' die het CNN kan identificeren:
- een oor
- vacht
- een oog
- een poot

Het probleem met dit soort modellen is dat men zeer grote aantallen data nodig heeft (miljoenen afbeeldingen) voor men ook maar in de buurt kan komen van de capaciteiten van het menselijk zicht. Bijvoorbeeld als een model getrained wordt met een afbeelding van een eend en het krijgt slechts 1 afbeelding van een eend; dan zal het model ook de achtergrondkleur en belichting als essentiele informatie van de foto beschouwen. Wanneer men een afbeelding van een andere kleur eend met andere belichting aanbiedt, zal het model geen idee hebben dat dit ook een eend is. De enige manier om dit probleem op te lossen is door grote hoeveelheden verschillende foto's van eenden in de gelabelde dataset te steken met verschillende achtergronden, lichtinvallen, kleuren,... Al deze afbeeldingen moeten correct gelabeld zijn.

Na miljoenen jaren natuurlijke evolutie zijn we nu op een interessant punt waarbij computers stilletjesaan de menselijke herkenningsskills kunnen matchen. In de toekomst zal hoogstwaarschijnlijk computer vision accurater worden dan human vision.

\subsubsection{\IfLanguageName{dutch}{Recurrent neural network}{Recurrent neural network}}
\label{sec:recurrent-neural-network}
Wanneer men videobeelden wilt analyseren kan men in essentie iedere frame nemen en gewoon analyseren met CNN zoals bij een afbeelding. Het probleem hierbij is echter dat iedere afbeelding op zich wordt geanalyseerd en dat het CNN geen weet heeft van de relatie tussen de vorige en de volgende frames. Hierdoor kan belangrijke context-informatie verloren geraken die gebruikt kan (en moet) worden om correcte voorspellingen te maken.
Om dit probleem op te lossen feed men de output van het CNN in een 'temporally sensitive model' ofwel een Recurrent Neural Network (RNN).

In tegenstelling to CNN kan RNN informatie onthouden over de vorige reeds geidentificeerd pixels en dit gebruiken bij het maken van toekomstige beslissingen.

\section{\IfLanguageName{dutch}{Archiveren van foto's}{Foto-archiving}}
\label{sec:archiveren-van-fotos}
...