%%=============================================================================
%% Voorwoord
%%=============================================================================

\chapter*{\IfLanguageName{dutch}{Woord vooraf}{Preface}}
\label{ch:voorwoord}

%% TODO:
%% Het voorwoord is het enige deel van de bachelorproef waar je vanuit je
%% eigen standpunt (``ik-vorm'') mag schrijven. Je kan hier bv. motiveren
%% waarom jij het onderwerp wil bespreken.
%% Vergeet ook niet te bedanken wie je geholpen/gesteund/... heeft

Deze bachelorproef kwam tot stand als laatste uitdaging bij het voltooien van de opleiding Toegepaste Informatica, afstudeerrichting Mobile Apps. Als werkstudent en developer kwam ik via mijn job bij Docbyte de laatste jaren uitvoerig in contact met cloud platformen. Bij Docbyte worden de cloud tools ingeschakeld om documenten te lezen en nuttige informatie te extracten via computer vision modellen. Het viel mij meteen op dat deze modellen goed werkten voor tekst-extractie; ik wilde graag weten of ze ook goed werkten voor afbeeldingen.

De paper had niet tot stand kunnen komen zonder de hulpen van verschillende mensen. In de eerste plaats wil ik mijn promotor Guy Dekoning bedanken voor de correcte opvolging en boeiende feedback. Meneer Dekoning heeft steeds zijn best gedaan om meetings in te passen in mijn druk schema als werkstudent. Verder bedank ik graag mijn teamlead en co-promotor Timon Devos, Timon heeft mij over de jaren veel bijgeleerd en is in het proces een vriend geworden. Ook bedank ik graag Docbyte waar ik 3 jaar gewerkt heb en mijn stage heb gedaan, ze boden mij de kans om door te groeien binnen het bedrijf en waren steeds ondersteunend tijdens examenperiodes. 

Tot slot bedank ik het Archief Gent en de medewerkers van de Beeldbank\footnote{https://beeldbank.stad.gent/index.php} voor de interessante feedback en het voorzien van de testdata. De Beeldbank is een fascinerende bron van oude en nieuwe foto's, ik raad de lezers van deze paper aan om zeker eens een kijkje te nemen.