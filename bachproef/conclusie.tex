%%=============================================================================
%% Conclusie
%%=============================================================================

\chapter{Conclusie}
\label{ch:conclusie}

% TODO: Trek een duidelijke conclusie, in de vorm van een antwoord op de
% onderzoeksvra(a)g(en). Wat was jouw bijdrage aan het onderzoeksdomein en
% hoe biedt dit meerwaarde aan het vakgebied/doelgroep? 
% Reflecteer kritisch over het resultaat. In Engelse teksten wordt deze sectie
% ``Discussion'' genoemd. Had je deze uitkomst verwacht? Zijn er zaken die nog
% niet duidelijk zijn?
% Heeft het onderzoek geleid tot nieuwe vragen die uitnodigen tot verder 
%onderzoek?

Zoals besproken in het vorige hoofdstuk wordt het duidelijk dat zowel Google Vision als AWS Rekognition bruikbaar zijn om de probleemstelling aan te pakken. De API's kunnen op snel tempo, met een hoge mate van zekerheid, labels aan verschillende soorten afbeeldingen toewijzen. De setup en configuratie is minimaal en kan door een gemiddeld bedrijf opgezet worden. Idealiter wordt er een front-end rond een of meerdere van deze API's gebouwd en kan een gewone PC gebruiker automatische labeling laten uitvoeren. Deze Bachelorproef kan als leidraad dienen voor organisatie's die foto-archieven hebben en deze op een correcte en gemakkelijke manier willen labelen. 
De imagga API scoorde niet voldoende hoog genoeg gebruikt te worden voor deze doelstellingen.

De labeling door API's heeft een hoge mate van correctheid (\char`\~ 90\%), toch schieten de API's momenteel op 1 vlak duidelijk te kort: het ontbreken van context. Context zorgt er voor dat een persoon aan de hand van de volledige informatie vervat in de afbeelding correct kan inschatten welke labels er toepasbaar zijn. De computer vision API's ontbreken deze context volledig en geven bijgevolg regelmatig labels die te algemeen zijn en weinig waarde toevoegen. Het is dan ook belangrijk om te benadrukken dat de API's besproken in deze studie manuele classificatie niet kunnen vervangen, het is interessanter om ze als een hulpmiddel te zien die archivarissen kunnen helpen om foto's sneller te labelen. Aangezien de API's een foutmarge van \char`\~ 10\% hebben is het belangrijk dat een persoon de gevonden labels controleert.

Na eerdere ervaringen met computer vision modellen werd deze uitkomst verwacht, het is duidelijk dat AI momenteel veel kan maar nog niet op het niveau van de menselijke kennis staat. Het laatste decennium heeft computer vision grote sprongen voorwaarts gemaakt, het zou dan ook interessant om een soortegelijk onderzoek opnieuw uit te voeren binnen 5 a 10 jaar. Verder onderzoek kan zich ook focussen op het dieper onderzoeken van de zekerheidsgraad threshold; welke threshold zorgt er voor de hoogste scores per API.