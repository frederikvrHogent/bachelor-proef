%%=============================================================================
%% Inleiding
%%=============================================================================

\chapter{\IfLanguageName{dutch}{Inleiding}{Introduction}}
\label{ch:inleiding}

Waarom dit onderwerp, waarom kan het interessant zijn

\section{\IfLanguageName{dutch}{Probleemstelling}{Problem Statement}}
\label{sec:probleemstelling}

Waarom is het nuttig => stad Gent => hebben vaak niet de kennis of resources om zelf machinelearning modellen op te zetten.

- needs further clean-up -
Dit brengt ons bij een belangrijk punt, en de reden dat deze bachelorproef geschreven wordt. Voor de meeste organisatie's is het niet realistisch om ten eerste miljoenen afbeeldingen te verzamelen en manueel te labelen en ten tweede de technische kennis in huis te hebben om dit alles te managen. Daarom geldt bij image-classificatie vaak de regel 'bigger is better', hoe groter de achterliggende dataset, hoe accurater de voorspellingen. 
Net om die reden zijn er slechts enkele engines beschikbaar die met een grote accuraatheid voorspellingen kunnen maken over algemene foto's.

\section{\IfLanguageName{dutch}{Onderzoeksvraag}{Research question}}
\label{sec:onderzoeksvraag}

Kunnen pre-trained machine learning engines gebruikt worden voor het archiveren van ongeclassificeerde foto's, zonder dat er technische kennis nodig is.

\section{\IfLanguageName{dutch}{Onderzoeksdoelstelling}{Research objective}}
\label{sec:onderzoeksdoelstelling}

Via een proof-of-concept wordt er hopelijk bewezen dat pre-trained machine learning nuttig kan zijn bij het classificeren van foto's.

\section{\IfLanguageName{dutch}{Opzet van deze bachelorproef}{Structure of this bachelor thesis}}
\label{sec:opzet-bachelorproef}

% Het is gebruikelijk aan het einde van de inleiding een overzicht te
% geven van de opbouw van de rest van de tekst. Deze sectie bevat al een aanzet
% die je kan aanvullen/aanpassen in functie van je eigen tekst.

De rest van deze bachelorproef is als volgt opgebouwd:

In Hoofdstuk~\ref{ch:stand-van-zaken} wordt een overzicht gegeven van de stand van zaken binnen het onderzoeksdomein, op basis van een literatuurstudie.

In Hoofdstuk~\ref{ch:methodologie} wordt de methodologie toegelicht en worden de gebruikte onderzoekstechnieken besproken om een antwoord te kunnen formuleren op de onderzoeksvragen.

% TODO: Vul hier aan voor je eigen hoofstukken, één of twee zinnen per hoofdstuk

In Hoofdstuk~\ref{ch:conclusie}, tenslotte, wordt de conclusie gegeven en een antwoord geformuleerd op de onderzoeksvragen. Daarbij wordt ook een aanzet gegeven voor toekomstig onderzoek binnen dit domein.